\documentclass[a4paper,12pt]{article}
\usepackage{setspace}
\doublespacing
\usepackage[backend=biber,style=apa]{biblatex}
\addbibresource{References.bib}
\usepackage{sectsty}
\usepackage{siunitx}
\usepackage{graphicx}
\usepackage[a4paper, total={3in, 9in}, textwidth=16cm,bottom=1in,top=1.4in]{geometry}
\usepackage{xcolor}
\usepackage{amsmath}
\usepackage{esvect}
\usepackage{amsthm}
\usepackage{hyperref}
\usepackage{float}
\usepackage{amssymb}
\usepackage{outlines}
\usepackage{caption}
\usepackage{subcaption}
\usepackage{esdiff}
\usepackage{setspace}
\newtheorem{lemma}{Lemma}
\newtheorem{proposition}{Proposition}
\doublespacing
\newcommand{\RNum}[1]{\uppercase\expandafter{\romannumeral #1\relax}}
\let\oldsi\si
\renewcommand{\si}[1]{\oldsi[per-mode=reciprocal-positive-first]{#1}}
\usepackage{enumitem}
\newcommand{\subtitle}[1]{%
  \posttitle{%
    \par\end{center}
    \begin{center}\large#1\end{center}
    \vskip0.5em}%
}
\newcommand{\degsym}{^{\circ}}
\newcommand{\Mod}[1]{\ (\mathrm{mod}\ #1)}
\usepackage{hyperref}
\hypersetup{
  colorlinks,
  citecolor=black,
  filecolor=black,
  linkcolor=black,
  urlcolor=black
}
\newcommand{\lb}{\\[8pt]}
\newenvironment*{cell}[1][]{\begin{tabular}[c]{@{}c@{}}}{\end{tabular}}
\newcommand{\img}[4]{\begin{center}
  \begin{figure}[H]
    \centering
    \includegraphics[width=#2\textwidth]{#1}
    \caption{#3}
    \label{fig:#4}
  \end{figure}
\end{center}}
\newcommand{\doubleimg}[4]{\begin{center}
  \begin{figure}[H]
    \centering
    \begin{subfigure}{.45\textwidth}
      \centering
      \includegraphics[width=1\linewidth]{#1}
      \caption{#2}
      \label{fig:sub1}
    \end{subfigure}
    \begin{subfigure}{.45\textwidth}
      \centering
      \includegraphics[width=1\linewidth]{#3}
      \caption{#4}
      \label{fig:sub2}
    \end{subfigure}
  \end{figure}
\end{center}}
\usepackage{fancyhdr}
\fancyfoot{}
\newcommand{\vect}[3]{\begin{bmatrix}
  #1 \\
  #2 \\
  #3
\end{bmatrix}}
\fancypagestyle{fancy}{\fancyfoot[R]{\vspace*{1.5\baselineskip}\thepage}}
\renewcommand{\contentsname}{Table of Contents}
\newcommand{\angled}[1]{\langle{#1}\rangle}
\newcommand{\paren}[1]{\left(#1\right)}
\newcommand{\sqb}[1]{\left[#1\right]}
\newcommand{\coord}[3]{\angled{#1,\, #2,\, #3}}
\newcommand{\pair}[2]{\paren{#1,\, #2}}
\usepackage{cleveref}
\crefname{lemma}{Lemma}{Lemmas}
\crefname{proposition}{Proposition}{Propositions}
\setlength{\headheight}{15pt}

\begin{document}


\pagenumbering{arabic}
\pagestyle{fancy}


\begin{titlepage}
  \begin{center}
    \vspace*{3cm}

    \textbf{\Large  {What is the relationship between the damping coefficient of a spring-mass oscillator submerged in different liquids and the densities of the liquids?}}

    \vspace{1cm}
    \large{Physics HL}\\
    \large{Internal Assessment}


    \vfill

    \vspace{1.5cm}

    Word count: TBD

  \end{center}
\end{titlepage}
\pagebreak
\tableofcontents
\pagebreak

\clearpage
\setcounter{page}{1}
\addtocontents{toc}{\protect\thispagestyle{empty}}

\section{Introduction}
This essay extends the investigation of simple harmonic motion by studying the damping coefficient and force of a damped oscillator submerged in water. Controlling damping through density is important in real-life systems ranging from shock absorbers to the stabilization of automobiles. The setup of the experiment consists of mainly a spring-mass oscillator submerged in a glass cylinder of liquid.


\subsection{The Research Question}
How Accurately Can Stoke's Law Estimate the Damping Coefficient of a Spherical Spring-Mass Simple Harmonic Oscillator Submerged in a Liquid?
\subsection{Background Information}
\subsection{Hypothesis}
\subsection{Variables}

\section{Main Dody}

\subsection{Data Collection}

\subsubsection{Apparatus and Materials}

\subsubsection{Procedures and Reproducing the Experiment}

\subsubsection{Risk Assessment}

\subsection{Data Processing}

\subsection{Data Analysis}

\subsubsection{Uncertainty Analysis}

\section{Conclusion}

\subsection{Evaluation}

\subsection{Extensibility}

\pagebreak

% \printbibliography

\end{document}