\documentclass[a4paper,12pt]{article}
\usepackage[a4paper, total={3in, 9in}, textwidth=17cm,bottom=1in,top=1.4in]{geometry}
\usepackage[raggedrightboxes]{ragged2e}
\usepackage{setspace}
\doublespacing
\usepackage[backend=biber,style=apa]{biblatex}
\addbibresource{Physics.bib}
\usepackage{sectsty}
\usepackage{siunitx}
\usepackage{graphicx}
\usepackage{longtable}
\usepackage[table,dvipsnames]{xcolor}
\usepackage{amsmath}
\usepackage{esvect}
\usepackage{amsthm}
\usepackage{outlines}
\usepackage{hyperref}
\usepackage{array} % for defining a new column type
\usepackage{varwidth} %for the varwidth minipage environment
\usepackage{float}
\usepackage{amssymb}
\usepackage{outlines}
\usepackage{caption}
\usepackage{subcaption}
\usepackage{esdiff}
\usepackage{setspace}
\usepackage[nameinlink]{cleveref}
\newtheorem{lemma}{Lemma}
\newtheorem{proposition}{Proposition}
\newtheorem{assumption}{Assumption}
\newtheorem{definition}{Definition}
\doublespacing
\newcommand{\RNum}[1]{\uppercase\expandafter{\romannumeral #1\relax}}
\let\oldsi\si
\renewcommand{\si}[1]{\oldsi[per-mode=reciprocal-positive-first]{#1}}
\usepackage{enumitem}
\newcommand{\subtitle}[1]{%
  \posttitle{%
    \par\end{center}
    \begin{center}\large#1\end{center}
    \vskip0.5em}%
}
\newcommand{\degsym}{^{\circ}}
\newcommand{\Mod}[1]{\ (\mathrm{mod}\ #1)}
\usepackage{hyperref}
\hypersetup{
  colorlinks,
  citecolor=black,
  filecolor=black,
  linkcolor=blue,
  urlcolor=blue
}
\newcommand{\lb}{\\[8pt]}
\newenvironment*{cell}[1][]{\begin{tabular}[c]{@{}c@{}}}{\end{tabular}}
\newcommand{\img}[4]{\begin{center}
  \begin{figure}[H]
    \centering
    \includegraphics[width=#2\textwidth]{#1}
    \caption{#3}
    \label{fig:#4}
  \end{figure}
\end{center}}
\newcommand{\doubleimg}[4]{\begin{center}
  \begin{figure}[H]
    \centering
    \begin{subfigure}{.45\textwidth}
      \centering
      \includegraphics[width=1\linewidth]{#1}
      \caption{#2}
      \label{fig:sub1}
    \end{subfigure}
    \begin{subfigure}{.45\textwidth}
      \centering
      \includegraphics[width=1\linewidth]{#3}
      \caption{#4}
      \label{fig:sub2}
    \end{subfigure}
  \end{figure}
\end{center}}
\usepackage{fancyhdr}
\fancyfoot{}
\newcommand{\vect}[3]{\begin{bmatrix}
  #1 \\
  #2 \\
  #3
\end{bmatrix}}
\fancypagestyle{fancy}{\fancyfoot[R]{\vspace*{1.5\baselineskip}\thepage}}
\renewcommand{\contentsname}{Table of Contents}
\newcommand{\angled}[1]{\langle{#1}\rangle}
\newcommand{\paren}[1]{\left(#1\right)}
\newcommand{\sqb}[1]{\left[#1\right]}
\newcommand{\coord}[3]{\angled{#1,\, #2,\, #3}}
\newcommand{\pair}[2]{\paren{#1,\, #2}}
\newcommand{\thcolor}{\cellcolor{Blue!25}}
\newcommand{\chcolor}{\cellcolor{RedOrange!25}}
\setlength{\headheight}{15pt}
\newcolumntype{P}[1]{>{\centering\arraybackslash}p{#1}}
\setlength{\belowcaptionskip}{-10pt}

\usepackage{cleveref}
\crefname{lemma}{Lemma}{Lemmas}
\crefname{assumption}{Assumption}{Assumptions}
\crefname{proposition}{Proposition}{Propositions}
\crefname{definition}{Definition}{Definitions}


\begin{document}


\pagenumbering{arabic}
\pagestyle{fancy}


\begin{titlepage}
  \begin{center}
    \vspace*{3cm}

    {\textbf{\Large{What is the relationship between the mass of a damped spring-block oscillator \{52.48g, 60.94g, 69.72g, 86.51g, 93.47g, 113.27g\} and the damping ratio?}}}



    \vspace{4cm}

    Word count: 2140

  \end{center}
\end{titlepage}
\pagebreak
\tableofcontents
\pagebreak

\clearpage
\setcounter{page}{1}
\addtocontents{toc}{\protect\thispagestyle{empty}}

\section{Introduction}
This essay extends the investigation of simple harmonic motion by studying the damping force of a damped oscillator submerged in water, aiming to scrutinize the relationship between the mass of the block and the damping ratio. By scrutinizing this relationship, the study aims to offer valuable insights that can inform the design of systems seeking to optimize damping levels for safety-related objectives.


\subsection{The Research Question}
What is the relationship between the mass of a damped spring-block oscillator \{52.48g, 60.94g, 69.72g, 86.51g, 93.47g, 113.27g\} and the damping ratio?

\subsection{Background Information}
An ideal and rather theoretical spring-mass system oscillates indefinitely, producing an ongoing sine or cosine curve. In reality, there will be a damping force that can be as minimally consequential as air resistance or as observable as viscous drag in a liquid. Anyway, energy is dissipated to the surroundings and hence the amplitude will gradually decrease until the oscillation stops. The extent to which the viscous drag force, modeled by Stoke's law, diminishes the oscillatory motion submerged in water depends on the mass, radius, and velocity of the oscillator, and the viscosity of the liquid at a given temperature. This investigation delves into the relationship between the independent variable of mass $m$, and the dependent variable of the damping ratio $\zeta$. The mass set used in the system consists of spheres of equal radii but different densities.

Stoke's Law determines the drag force acting upon an object traveling through a fluid -- proportional to the object's velocity and is the following per the Data Booklet, in Newtons
$$F_d = 6\pi r \eta v $$ where
\begin{itemize}
  \item $r$ (m) is the radius of the sphere
  \item $\eta$ $(\si{\N\s\per\m\squared})$ is the viscosity of the fluid at some temperature
  \item $v$ $(\si{\m\per\s})$ is the velocity of the object in the fluid
\end{itemize}
This holds if and only if the object's speed is low such that the flow is \textit{laminar}, and the object is \textit{spherical}.

\begin{assumption}
  \label{as:1}
  The damping force is the viscous drag force.
\end{assumption}

Moreover, the motion of the oscillator is modeled by the following differential equation, with the damping force proportional to velocity \parencite{miller_2004_13}:
\begin{equation}
  \label{eq:1}
  m\ddot{x}(t) + b\dot{x}(t) + kx(t) = 0
\end{equation}
$m$ denotes the mass of the oscillator, $b$ denotes the damping coefficient that is constant in a system of constant fluid viscosity, and $k$ represents the spring constant. The damping ratio $\zeta$ is defined as the ratio of the damping coefficient to the critical damping coefficient --- the damping coefficient when the system returns to equilibrium without completing an oscillation. That is $$\zeta = \frac{b}{c_c} = \frac{b}{2mw_0}$$
where $w_0$ is the natural frequency of the system
\begin{align*}
  w_0            = \sqrt{\frac{k}{m}}\implies \zeta & = \frac{b}{2\sqrt{mk}}
\end{align*}

\subsection{Hypothesis}

Suppose that the constant of proportionality of the damping force with speed is given by $b = 6\pi r \eta$, per Stoke's law. Then
\begin{align*}
  6\pi r \eta            & = 2\zeta \sqrt{mk}            \\
  9\pi^2 r^2 \eta^2      & = \zeta^2mk                   \\
  \frac{\zeta^2}{m^{-1}} & = \frac{9\pi^2 r^2 \eta^2}{k}
\end{align*}
\begin{proposition}
  For a fixed sphere radius $r$ and viscosity $\eta$, \begin{equation}
    \zeta^2 \propto m^{-1}
  \end{equation}
  Equivalently, the graph $\zeta^2/m^{-1}$ is ideally a straight line, although not necessarily passing through the origin, as some form of systematic errors are expected.
  \label{prop:1}
\end{proposition}

\cref{prop:1} has a few dependencies that may affect its accuracy. It relies on \cref{as:1}, which, in turn, requires that the flow is laminar. This is a reasonable assumption given that the spheres are relatively small, and the speed is low; the potential for turbulence is insignificant and even if present, the impact is minimal. Furthermore, the buoyancy force on the object
$$F_b = \rho g V$$ where
\begin{itemize}
  \item $\rho$ is the density of the fluid
  \item $V$ is the volume of the object
  \item $g$ is the gravitational field strength
\end{itemize}
is always present but is not modeled in the differential equation at \cref{eq:1}. Notwithstanding, the absence of considerations of buoyancy has negligible impact on precision. The setup involves spheres of radius $2.5\si{\cm}$ and water of density $997\si{\kg\per\m\cubed}$, then
$$F_b \approx 10 \times 1000 \times \frac{4}{3}\pi (2.5\times 10^{-2})^3 < 1\si{\N}$$
\underline{Null hypothesis ($H_0$)}: There is no relationship between the mass and the damping ratio.
\underline{Alternative hypothesis ($H_1$)}: There is a relationship between the mass and the damping ratio.
\pagebreak

\section{Research Design}

\subsection{Variables}

\begin{center}
  \begin{tabular}{|p{0.14\textwidth}|p{0.11\textwidth}|p{0.33\textwidth}|p{0.3\textwidth}|}
    \hline
    \multicolumn{2}{|c}{\thcolor Variables} & \multicolumn{1}{|c}{\thcolor Explanation} & \multicolumn{1}{|c|}{\thcolor Measurement}                                                                                                                                \\  \hline
    Independent                             & Mass $m$                                  & The masses of the spherical metal bobs                                              & Measured using an electric balance, to two d.p. with uncertainty $\pm 0.01 \si{\g}$ \\
    \hline
    Dependent                               & Peak Amplitude                            & 3 peaks will  be measured. These data points are used to compute the damping ratio. & Ruler, with uncertainty $\pm 0.1\si{\cm}$                                           \\
    \hline
  \end{tabular}
  \label{tab:1}
  \captionof{table}{Variables of investigation}
\end{center}

\begin{center}
  \begin{tabular}{|p{0.3\textwidth}|p{0.3\textwidth}|p{0.3
    \textwidth}|}

    \hline
    \multicolumn{3}{|c|}{\thcolor Control Variables}                                                                                                                                                                                                                                                                                                                      \\ \hline
    \multicolumn{1}{|c|}{Variable}                        & \multicolumn{1}{|c|}{Rationale}                                     & \multicolumn{1}{|c|}{Means of Control}                                                                                                                                                                                                  \\ \hline
    Liquid viscosity   $0.85$ $(\si{\N\s\per\m\squared})$ & To keep the drag coefficient per Stoke's Law constant               & Using \textbf{water from the same source} throughout the experiment, and keep the \textbf{room temperature} constant by turning off the air conditioning. In fact, during the experiment, the room temperature was $27\degsym \text{C}$ \\ \hline
    Radius of the sphere, $r = 2.5\si{\cm}$               & This is so that the drag force, per Stoke's law, is constant        & Using spherical masses of the same diameter                                                                                                                                                                                             \\ \hline
    Spring constant, $k = 100\si{\N\per\m}$               & This is so that the natural frequency of the system is constant     & Using the same spring throughout the experiment. Read from the label on the box.                                                                                                                                                        \\ \hline
    Initial amplitude, $A_0 = 5\si{\cm}$                  & So that the damping ratio is not affected by the initial conditions & Pulling the spring down by equal distances for every mass.  Measured by a ruler with $\pm 0.1\si{\cm}$                                                                                                                                  \\ \hline
    Damping coefficient                                   & Linearizes relationship between damping ratio and mass              & Previous means of control keep fluid viscosity and object dimensions constant, so the damping coefficient will be constant.                                                                                                             \\
    \hline
  \end{tabular}
  \label{tab:2}
  \captionof{table}{Control variables}
\end{center}


\pagebreak

\subsection{Apparatus and Materials}
\img{figs/setup.png}{0.8}{The experimental setup}{setup}
\begin{outline}[enumerate]
  \1 Ruler ($\pm 0.1\si{\cm}$)
  \1 Electric balance ($\pm 0.01\si{\g}$)
  \1 Spring with spring constant $k = 100\si{\N\per\m}$
  \1 Mass set with six spherical masses of diameter $2.5\si{\cm}$
  \1 Beaker
  \1 Tap water
  \1 Camera --- keeps a record for future reference
  \1 Thermometer --- to ensure that the liquid stays at a constant temperature
  \1 Retort stand
  \1 Clamp
\end{outline}

\subsection{Methodology}

\begin{outline}[enumerate]
  \1 Take a scrotch tape, and cut into segments of $8$ $\si{\cm}$ in length.
  \1 Arrange the masses in increasing order, i.e. $m_1 < m_2 < ... < m_n$, where $n$ is the number of masses in the mass set.
  \1 Fill in the beaker with $A$ $\si{\mL}$ of tap water
  \1 Measure the temperature of the water, and record as $T_0$
  \1 Attach the clamp to the retort stand.
  \1 Lean the ruler vertically against the retort stand.
  \1 Place the beaker directly under the clamp such that the clamp lies at the center of the beaker in bird's eye view.
  \1 For $i$ from 1 to $n$ inclusive, repeat the following
  \2 Take the $i$th mass, $m_i$, and attach its hook to the spring.
  \2 Take a piece of tape, wrap it as many times as possible across the point of connection.
  \2 Attach the spring-mass pair to the clamp and let the mass hang freely.
  \2 Adjust the height of the clamp so that the mass is submerged in the water.
  \2 Measure the height of the mass from the surface of the desk, and record as $h_0$
  \2 Adjust the vertical camera position such that it levels with the reading $(h_0 - \dfrac{A_0}{2}) \si{\cm}$ to minimize parallax error of the camera.
  \2 For $j$ from 1 to 5 inclusive, repeat the following
  \3 Pull the mass down by $A_0\,\si{\cm}$ by reading $h_0 - A_0$ from the ruler at eye level. The center of the mass should be the height a which readings are taken.
  \3 Start the camera recording.
  \3 Release the mass.
  \3 Record the clip until the 3rd negative peak is reached.
  \3 Save the clip as \verb|Mass_i_Trial_j.mp4|
  \3 Open the clip in a video viewer.
  \3 Identify the first 3 negative peaks, one in each cycle.
  \3 Record the respective heights above the desk as $\{h_k\}^{5}_{k=1}$.
  \3 Then the corresponding peak amplitudes are $A_k = h_0 - h_{k}$
  \2 Take the average values across the three trials for each peak amplitude.
  \2 Measure the water temperature and wait until it stays constant at $27\degsym \text{C}$.
\end{outline}

\subsection{Preliminary Trials}

Preliminary trials were conducted to identify and resolve potential challenges, as well as to refine the experimental conditions. As planned, lighter masses were used but during the experiment, more material was used to assemble heavier masses; this is because the trial proved that spheres with insufficient masses do not oscillate. Initially, no mechanism was utilized to attach the masses to the spring, which led to slipping. To improve the attachment's security and uniformity, Scotch tape was tested and found to be more effective, significantly reducing variability and enhancing setup repeatability. Precise measurement of the initial height ($h_0$) and consistent displacement of the mass by $A_0,\si{\cm}$ were also challenging. Aligning the ruler vertically proved crucial for accuracy. Adjustments to the clamp and careful ruler leveling minimized parallax error, further reduced by positioning the camera at eye level.

\subsection{Risk Assessment}

\begin{center}
  \begin{tabular}{|P{0.19\textwidth}|p{0.71\textwidth}|}
    \hline
    \thcolor Consideration & \multicolumn{1}{c|}{\thcolor {Relevance and Mitigation}         }                                                                                                                                                                                                                                   \\ \hline
    Safety                 & All valuable assets are kept away from the experimental setup to prevent accidental detachments of the spring-mass system leading to the mass being thrown out and causing damage. Moreover, the amplitude was kept low to prevent fierce movements that could break the beaker or computer camera. \\ \hline
    Ethical                & No use of animals or human bodies involved. However, the recording selects the location and angle that avoids the storage and exposure of any personal information or objects of the lab owner, ensuring maximum privacy protection.                                                                \\ \hline
    Environmental          & The entire experiment aims to reuse the same water throughout to minimize water wastage.    Furthermore, the spring was carefully pulled to prevent it from passing its elastic limit and becoming unusable for future experiments.                                                                 \\ \hline
  \end{tabular}
  \label{tab:3}
  \captionof{table}{Risk Assessment}
\end{center}

\pagebreak

\section{Results}

\subsection{Raw Data}

\subsubsection{Qualitative Data}
During the entire experiment, there is no spillage of water, which means that the volume is kept constant. Moreover, the spring is always able to return to its original length without any permanent deformation, indicating that the spring is not stretched beyond its elastic limit and that Hooke's Law applies throughout.

\subsubsection{Quantitative Data}

Let $m$ denote the mass of a bob, and $A_n$ denote the $n$-th peak height.

\begin{table}[htbp]
  \begin{center}
    \bgroup
    \renewcommand{\arraystretch}{1.5}
    \resizebox{0.98\textwidth}{!}{%
      \begin{tabular}{|m{0.076\textwidth}|*{16}{m{0.0625\textwidth}|}}
        \hline
        \multicolumn{1}{|m{0.076\textwidth}|}{\thcolor {$m \pm 0.01$ ($\si{\g}$)}} & \multicolumn{5}{l|}{\thcolor \Large$A_1 \pm 0.1$ ($\si{\cm}$)} & \multicolumn{5}{l|}{\thcolor \Large$A_2 \pm 0.1$ ($\si{\cm}$)} & \multicolumn{5}{l|}{\thcolor \Large$A_3 \pm 0.1$ ($\si{\cm}$)}                                                                         \\ \hline
        52.48                                                                      & 2.9                                                            & 2.9                                                            & 2.8                                                            & 2.8 & 2.9 & 1.7 & 1.8 & 1.7 & 1.7 & 1.7 & 1.0 & 1.0 & 1.0 & 1.1 & 1.0 \\ \hline
        60.94                                                                      & 2.9                                                            & 3.0                                                            & 3.0                                                            & 2.9 & 3.0 & 1.8 & 1.9 & 1.8 & 1.8 & 1.8 & 1.1 & 1.1 & 1.1 & 1.2 & 1.1 \\ \hline
        69.72                                                                      & 3.1                                                            & 3.1                                                            & 3.1                                                            & 3.1 & 3.1 & 1.9 & 2.0 & 1.8 & 1.9 & 1.9 & 1.2 & 1.2 & 1.2 & 1.3 & 1.2 \\ \hline
        86.51                                                                      & 3.3                                                            & 3.3                                                            & 3.3                                                            & 3.3 & 3.3 & 2.1 & 2.1 & 2.2 & 2.1 & 2.1 & 1.4 & 1.4 & 1.4 & 1.4 & 1.3 \\ \hline
        93.47                                                                      & 3.5                                                            & 3.3                                                            & 3.4                                                            & 3.3 & 3.5 & 2.4 & 2.2 & 2.4 & 2.2 & 2.2 & 1.5 & 1.5 & 1.5 & 1.5 & 1.5 \\ \hline
        113.27                                                                     & 3.5                                                            & 3.5                                                            & 3.7                                                            & 3.6 & 3.5 & 2.3 & 2.4 & 2.4 & 2.5 & 2.3 & 1.6 & 1.4 & 1.5 & 1.7 & 1.4 \\ \hline
      \end{tabular}}
    \label{tab:4}
    \captionof{table}{Raw data of peak amplitudes}
    \egroup
  \end{center}
\end{table}

\pagebreak

\subsection{Data Processing}

\subsubsection{Transformation}\label{subsubsec:transformation}

For the sake of consistency, let us start with the following definition.

\begin{definition}
  Let the tuple $(i, j)$ denote the pair of peaks at $i$ and $j$ respectively, with $i < j$.
\end{definition}

\noindent Using the \textit{logarithmic decrement} method per \Citeauthor{inman_2008_engineering} (\citeyear{inman_2008_engineering})
$$\delta_{(i, j)} = (j-i)\ln\paren{\frac{A_i}{A_{j}}} \text{ and } \delta = \frac{2\pi \zeta}{\sqrt{1 - \zeta^2}}$$ where $A_i$ and $A_j$ denote the peak height measurements at the $i$-th and $j$-th peaks respectively. It then follows that
\begin{align*}
  \delta^2                   & = \frac{4\pi^2 \zeta^2}{1 - \zeta^2}      \\
  \delta^2 - \delta^2\zeta^2 & = 4\pi^2\zeta^2                           \\
  \delta^2                   & = \zeta^2\paren{4\pi^2 + \delta^2}        \\
  \zeta                      & = \frac{\delta}{\sqrt{4\pi^2 + \delta^2}}
\end{align*}

$\zeta$, by definition, may be considered as a function of a pair of peaks. This function is applied to \textbf{every pair}, producing a set of values of $\zeta$, across which we will take the average to most accurately identify the damping ratio --- computing $\zeta$ from a mere peak-pair poses vulnerability to random error and lacks reliability. This step is encapsulated in the Python snippet attached in the appendix section at \cref{app:1}, which belongs to an overall script for processing the data and generating the graph.

Note that $\overline{A_i}$ denotes the average value of the $i$-th peak across the five repetitions.

\begin{table}[htbp]
  \begin{center}
    \bgroup
    \renewcommand{\arraystretch}{1.5}
    \begin{tabular}{|P{0.20185\textwidth}|*{3}{P{0.135\textwidth}|}P{0.205\textwidth}|}
      \hline
      \thcolor {$m \pm 0.01$ ($10^{-3}\si{\kg}$)} & \thcolor $\overline{A_1}$ ($10^{-2}\si{\m}$) & \thcolor $\overline{A_2}$ ($10^{-2}\si{\m}$) & \thcolor $\overline{A_3}$ ($10^{-2}\si{\m}$) & \thcolor $\zeta^2/10^{-3}$ (unitless) \\ \hline
      52.48                                       & 2.9                                          & 1.7                                          & 1.0                                          & 7.13                                  \\ \hline
      60.94                                       & 3.0                                          & 1.8                                          & 1.1                                          & 6.33                                  \\ \hline
      69.72                                       & 3.1                                          & 1.9                                          & 1.2                                          & 5.67                                  \\ \hline
      86.51                                       & 3.3                                          & 2.1                                          & 1.4                                          & 4.63                                  \\ \hline
      93.47                                       & 3.4                                          & 2.3                                          & 1.5                                          & 4.22                                  \\ \hline
      113.27                                      & 3.6                                          & 2.4                                          & 1.5                                          & 4.83                                  \\ \hline
    \end{tabular}
    \label{tab:5}
    \captionof{table}{Processed data}
    \egroup
  \end{center}
\end{table}

\pagebreak

\subsubsection{Uncertainty Analysis}

In this section, one will learn about how uncertainties propagate through the calculation of the damping ratio.\lb
We will adopt the following method for the computation of uncertainties: Let $\Delta x$ denote the absolute uncertainty of $x$. For $\Delta x \ll x$, the uncertainty of $f(x)$ is given by the following \parencite{vacher_2001_the}

\begin{equation}\label{eq:uncertainty}
  \Delta f(x) \approx \diff{f(x)}{x}\cdot \Delta x
\end{equation}

From the raw data of the peak heights to $\zeta$, as seen in the data transformation section at \cref{subsubsec:transformation}, the initial measurements have gone through a series of functions to finally arrive at a value for $\zeta$. Thus, to visualize the propagation of uncertainty, we will consider how the uncertainty is propagated in each individual step of the transformation linearly. The highlighted rows represent the final stages of the uncertainty propagation for the independent and dependent variables respectively.


Let $i$, $j$ be the indices of any pair of peaks such that $i < j$, and also let $\gamma_{i, j} = \dfrac{A_i}{A_{j}}$, the following table summarizes the uncertainties involved in the calculation of the damping ratio from a pair of peaks indexed with $(i, j)$. The right-most column is obtained by applying the procedure outlined by \cref{eq:uncertainty}.

\begin{table}[H]
  \begin{center}
    \bgroup
    \renewcommand{\arraystretch}{1.5}
    \begin{tabular}{|c|c|c|}
      \hline
      \thcolor {Variable}       & \thcolor Represented Transformation                   & \thcolor Propagated Absolute Uncertainty                                                          \\ \hline
      $m$                       &                                                       & $0.01\si{\g}$                                                                                     \\ \hline
      \chcolor $m^{-1}$         & \chcolor                                              & \chcolor $(m)^{-2}\cdot 0.01\si{\per\g}$                                                          \\ \hline
      $A_k,\,\forall k$         &                                                       & $0.1\si{\cm}$                                                                                     \\ \hline
      $\gamma_{i,j}$            & $\frac{A_i}{A_{j}}$                                   & $\Delta \gamma_{i, j} = \frac{A_i}{A_{j}}(\frac{\Delta A_i}{A_{i}} + \frac{\Delta A_{j}}{A_{j}})$ \\ \hline
      $\delta_{i,j}$            & $\frac{1}{j - i}\ln{\gamma_{i,j}}$                    & $\Delta \delta_{i,j} = \frac{1}{j - i}\times \frac{\Delta \gamma_{i,j}}{\gamma_{i,j}}$            \\ \hline
      $\zeta_{i, j}$            & $\frac{\delta_{i,j}}{\sqrt{4\pi^2 + \delta_{i,j}^2}}$ & $\Delta \zeta_{i, j} = \diff{\zeta}{\delta} \cdot \Delta\delta$                                   \\ \hline
      \chcolor $\zeta_{i, j}^2$ & \chcolor                                              & \chcolor $2(\zeta_{i, j})\cdot (\Delta \zeta_{i, j})$                                             \\ \hline
    \end{tabular}
    \label{tab:6}
    \captionof{table}{Uncertainty Propagation}
    \egroup
  \end{center}
\end{table}

\vspace{\baselineskip}
To implement the above logic in Python, the previous code snippet in \cref{fig:damping_code} under \cref{app:1} is extended to include the uncertainty calculations. See \cref{app:2} for the full code snippet.



Running the script again yields the following table of uncertainties:
\begin{table}[htbp]
  \begin{center}
    \bgroup
    \renewcommand{\arraystretch}{1.5}
    \begin{tabular}{|P{0.16\textwidth}|P{0.25\textwidth}|P{0.25\textwidth}|P{0.145\textwidth}|}
      \hline
      \thcolor {$\Delta m/10^{3}$ ($\si{\kg}$)} & \thcolor $m^{-1} (\si{\per\kg})$ & \thcolor $\Delta m^{-1} (\si{\per\kg})$ & \thcolor $\Delta \zeta^2/10^{-3}$ (unitless) \\ \hline
      52.48                                     & 19.23                            & 0.003698 $\approx$ 0.00                 & 3.35                                         \\ \hline
      60.94                                     & 16.39                            & 0.002687 $\approx$ 0.00                 & 2.53                                         \\ \hline
      69.72                                     & 14.29                            & 0.002041 $\approx$ 0.00                 & 2.71                                         \\ \hline
      86.51                                     & 11.49                            & 0.001321 $\approx$ 0.00                 & 2.42                                         \\ \hline
      93.47                                     & 10.75                            & 0.001156 $\approx$ 0.00                 & 1.92                                         \\ \hline
      113.27                                    & 8.850                            & 0.000783 $\approx$ 0.00                 & 2.01                                         \\ \hline
    \end{tabular}
    \label{tab:7}
    \captionof{table}{Absolute uncertainties}
    \egroup
  \end{center}
\end{table}

\subsection{Graphical Interpretation}

\img{../model/figures/graph.png}{1}{Graph of $\zeta^2$ against $m^{-1}$, with best-fit line and predicted line}{graph1}
\img{../model/figures/graph_max_min.png}{1}{Graph of $\zeta^2$ against $m^{-1}$ --- literature review}{graph2}

The two graphs above show the relationship between the damping ratio squared and the inverse of the mass from the resultant dataset, with error bars plotted. The first graph, \cref{fig:graph1}, shows a scatter plot of the data points from \cref{tab:5}, with the best-fit line and the line predicted using Stoke's Law stated in the hypothesis proposition section. The data points suggest a linear relationship between $\zeta^2$ and $m^{-1}$, with the Pearson correlation coefficient $r = 0.684$. The second graph, \cref{fig:graph2}, shows the prediction line in comparison to the maximum and minimum-slope lines resulting from the propagation of uncertainty --- indeed, the theoretical prediction does lie within the error bars of the data points.

\section{Conclusion}

This essay delved into the effect of mass $m$ on the damping ratio $\zeta$ of a damped spring-block oscillator submerged in water, with the hypothesis that $$\zeta^2 \propto m^{-1}$$ The results from the experiment suggest a linear relationship between the damping ratio squared and the inverse of the mass, with a Pearson correlation coefficient of 0.700, indicating a relatively strong linear trend. The theoretical prediction line falls within the interval of the maximum and minimum slopes due to the uncertainties in measurements, supporting the hypothesis that the damping ratio squared is inversely proportional to the mass, within the experimented range of 52.48 to 113.27 grams.

%wc: 1848%

The observation that the relationship is not perfectly linear is a consequence of the uncertainties that arise from the limitations of measuring devices and human errors. One primary source of random error is the unintentional but inevitable extra force applied when releasing the spring in the liquid. Nonetheless, the mass set did not produce anomalies and massive systematic errors, suggesting that the environment and the experimental setup were controlled effectively. \cref{prop:1} suggested that the expected gradient, $m_0$ would be \begin{align*}
  m_0 & =    \frac{9\pi^2r^2\eta^2}{k}               \\
      & =        \frac{9\pi^2(0.025)^2(0.85)^2}{100} \\
      & \approx  4.011 \times 10^{-4}
\end{align*}

The minimum and maximum slopes are $-3.339\times 10^{-4}$ and $9.109\times 10^{-4}$ respectively, with the experimental gradient in between. This shows that the drag force on the oscillator in water can be estimated using Stoke's Law. The error propagation does create rather significant error bars due to the already small values of $\zeta^2$, particularly for smaller masses. However, this does accept the theoretical value and is expected since $\zeta^2$ is very sensitive to a change in mass.

\subsection{Evaluation}

\subsubsection{Strengths}

\begin{center}
  \begin{tabular}{|p{0.3\textwidth}|p{0.6\textwidth}|}
    \hline
    \thcolor Strength                                   & \thcolor Observations                                                                                                                                                                                                                                                                                          \\ \hline
    Attempt to minimize random error in peak amplitudes & Across the three peaks, all possible pairs were used to calculate a value of $\zeta$, across all of them an average is taken. This produces the $\overline{\zeta}$ that is closer to the true value for each possible oscillation, as the average measurement is less susceptible to random forces introduced. \\ \hline
    Error minimization procedures                       & The experiment was designed to minimize random and systematic errors. Taking fix trials and computing the average builds upon the attempt to minimize error by averaging pairs across three peaks, further diminishing the impact of potential anomalies and outliers in the dataset.                          \\ \hline
    Consistency                                         & The entire experiment followed a sequence of algorithmic and rigid instructions to ensure that each trial was conducted in the same conditions.                                                                                                                                                                \\ \hline
    Preliminary tests                                   & Preliminary tests were conducted to identify potential challenges and refine the experimental conditions, ensuring that the final experiment was conducted under optimal conditions.                                                                                                                           \\ \hline
  \end{tabular}
\end{center}

\subsubsection{Weaknesses --- Random Errors}

\begin{center}
  \begin{tabular}{|p{0.15\textwidth}|p{0.35\textwidth}|p{0.35\textwidth}|}
    \hline
    \thcolor Weakness                                   & \thcolor Observations                                                                                                                                                                                                                & \thcolor Improvements                                                                                                                   \\ \hline
    Inaccuracy of manual identification from video clip & The manual identification of the peaks from the video clip may have introduced human error, leading to further inaccuracies in the peak amplitude measurements. These uncertainties may not have been captured by the data analysis. & Integrating computer vision can automatically identify the peaks would reduce human error and improve the accuracy of the measurements. \\ \hline
    Release of the oscillator                           & The release of the oscillator may have introduced an additional force, leading to inconsistencies in the peak amplitude measurements.                                                                                                & Using a tong, for instance, would ensure a consistent force and minimize the introduction of additional forces.                         \\ \hline
    Scrotch tape attachment                             & At times the attachment demonstrated signs of slipping or looseness, which may have affected the oscillation and effect of damping.                                                                                                  & Using a more secure attachment method, such swivel connectors, would ensure the mass' hook is firmly attached to the spring.            \\ \hline
  \end{tabular}
\end{center}

\pagebreak

\subsubsection{Weaknesses --- Systematic Errors}
\begin{center}
  \begin{longtable}{|p{0.15\textwidth}|p{0.35\textwidth}|p{0.35\textwidth}|}
    \hline
    \thcolor Weakness                     & \thcolor Observations                                                                                                                                                                                                                                                                                                                                        & \thcolor Improvements                                                                                                         \\ \hline
    Limited sample strength.              & The experiment only used five masses, which may not be sufficient to draw definitive conclusions. This was due to limited equipment provided by the laboratory. Moreover, it does not cover a great range of masses either, hence the model failed to inspect the behavior of the oscillatory motion for larger masses and validate the hypothesis for them. & Increasing the number of masses tested would provide a more comprehensive dataset and enhance the reliability of the results. \\ \hline
    Varying spring constant               & The spring constant was assumed to be constant, but in reality, it may have varied slightly with use, leading to inaccuracies in theoretical prediction of the gradient.                                                                                                                                                                                     & Preparing a set of identical springs and consistently switching between them would minimize fatigue.                          \\ \hline
    Camera positioning                    & The camera positioning may have introduced parallax errors, leading to systematic errors in all of the peak amplitude measurements.                                                                                                                                                                                                                          & Through calculations, find the optimal position to minimize the average uncertainty across the peak measurements.             \\ \hline
    Large uncertainty ranges in $\zeta^2$ & The error spans in $\zeta^2$ were large, particularly for smaller masses.                                                                                                                                                                                                                                                                                    & Using more accurate measuring devices, such as even more accurate rulers and electric balances.                               \\ \hline
  \end{longtable}
\end{center}

\subsubsection{Extensions}

A possible extension of this experiment involves investigating the effect of different fluid densities on the damping ratio of the oscillating system. By repeating the experiment with fluids of known densities (e.g., water, glycerol, oil) whose values can be found on the Internet, and keeping control variables constant, i.e. mass, initial amplitude, spring constant, and temperature, one can analyze how changing densities influence the damping behavior. As a hypothesis, the higher the density is, the larger the viscous drag force, and thus the greater the damping ratio.

Alternatively, one may also explore the impact of varying the spring constant on the damping ratio. By using springs varying in stiffness, one can investigate how the damping ratio changes with the spring constant. The hypothesis, based on \cref{prop:1}, would be that the higher the spring constant, the lower the damping ratio, as the spring would exert a greater force to counteract the damping force.

\pagebreak

\printbibliography[
  heading=bibintoc,
  title={Bibliography}
]

\pagebreak

\section{Appendix}

\subsection{Python Script for Data Processing}\label{app:1}
\img{figs/code/damping.png}{0.95}{Calculating the damping ratio with pandas in Python}{damping_code}

\pagebreak

\subsection{Extended Script with Uncertainty Propagation}\label{app:2}

\img{figs/code/unc.png}{0.75}{Calculating the damping ratio with uncertainties in Python}{unc_code}

\end{document}