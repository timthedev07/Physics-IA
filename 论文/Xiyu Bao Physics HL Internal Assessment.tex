\documentclass[a4paper,12pt]{article}
\usepackage{setspace}
\doublespacing
\usepackage[backend=biber,style=apa]{biblatex}
\addbibresource{Physics.bib}
\usepackage{sectsty}
\usepackage{siunitx}
\usepackage{graphicx}
\usepackage[a4paper, total={3in, 9in}, textwidth=16cm,bottom=1in,top=1.4in]{geometry}
\usepackage[table,dvipsnames]{xcolor}
\usepackage{amsmath}
\usepackage{esvect}
\usepackage{amsthm}
\usepackage{outlines}
\usepackage{hyperref}
\usepackage{array} % for defining a new column type
\usepackage{varwidth} %for the varwidth minipage environment
\usepackage{float}
\usepackage{amssymb}
\usepackage{outlines}
\usepackage{caption}
\usepackage{subcaption}
\usepackage{esdiff}
\usepackage{setspace}
\newtheorem{lemma}{Lemma}
\newtheorem{proposition}{Proposition}
\newtheorem{assumption}{Assumption}
\doublespacing
\newcommand{\RNum}[1]{\uppercase\expandafter{\romannumeral #1\relax}}
\let\oldsi\si
\renewcommand{\si}[1]{\oldsi[per-mode=reciprocal-positive-first]{#1}}
\usepackage{enumitem}
\newcommand{\subtitle}[1]{%
  \posttitle{%
    \par\end{center}
    \begin{center}\large#1\end{center}
    \vskip0.5em}%
}
\newcommand{\degsym}{^{\circ}}
\newcommand{\Mod}[1]{\ (\mathrm{mod}\ #1)}
\usepackage{hyperref}
\hypersetup{
  colorlinks,
  citecolor=black,
  filecolor=black,
  linkcolor=black,
  urlcolor=black
}
\newcommand{\lb}{\\[8pt]}
\newenvironment*{cell}[1][]{\begin{tabular}[c]{@{}c@{}}}{\end{tabular}}
\newcommand{\img}[4]{\begin{center}
  \begin{figure}[H]
    \centering
    \includegraphics[width=#2\textwidth]{#1}
    \caption{#3}
    \label{fig:#4}
  \end{figure}
\end{center}}
\newcommand{\doubleimg}[4]{\begin{center}
  \begin{figure}[H]
    \centering
    \begin{subfigure}{.45\textwidth}
      \centering
      \includegraphics[width=1\linewidth]{#1}
      \caption{#2}
      \label{fig:sub1}
    \end{subfigure}
    \begin{subfigure}{.45\textwidth}
      \centering
      \includegraphics[width=1\linewidth]{#3}
      \caption{#4}
      \label{fig:sub2}
    \end{subfigure}
  \end{figure}
\end{center}}
\usepackage{fancyhdr}
\fancyfoot{}
\newcommand{\vect}[3]{\begin{bmatrix}
  #1 \\
  #2 \\
  #3
\end{bmatrix}}
\fancypagestyle{fancy}{\fancyfoot[R]{\vspace*{1.5\baselineskip}\thepage}}
\renewcommand{\contentsname}{Table of Contents}
\newcommand{\angled}[1]{\langle{#1}\rangle}
\newcommand{\paren}[1]{\left(#1\right)}
\newcommand{\sqb}[1]{\left[#1\right]}
\newcommand{\coord}[3]{\angled{#1,\, #2,\, #3}}
\newcommand{\pair}[2]{\paren{#1,\, #2}}
\newcommand{\thcolor}{\cellcolor{Blue!25}}
\usepackage{cleveref}
\crefname{lemma}{Lemma}{Lemmas}
\crefname{assumption}{Assumption}{Assumptions}
\crefname{proposition}{Proposition}{Propositions}
\setlength{\headheight}{15pt}
\newcolumntype{P}[1]{>{\centering\arraybackslash}p{#1}}


\begin{document}
\setlength{\belowcaptionskip}{-40pt}


\pagenumbering{arabic}
\pagestyle{fancy}


\begin{titlepage}
  \begin{center}
    \vspace*{3cm}

    {\textbf{\Large{What is the relationship between the mass of a damped spring-block oscillator {23.42g, 35.38g, 52.48g, 69.72g, 93.47g} and the damping ratio?}}}



    \vspace{4cm}

    Word count: TBD

  \end{center}
\end{titlepage}
\pagebreak
\tableofcontents
\pagebreak

\clearpage
\setcounter{page}{1}
\addtocontents{toc}{\protect\thispagestyle{empty}}

\section{Introduction}
This essay extends the investigation of simple harmonic motion by studying the damping force of a damped oscillator submerged in water, aiming to scrutinize the relationship between the mass of the block and the damping ratio. By scrutinizing this relationship, the study aims to offer valuable insights that can inform the design of systems seeking to optimize damping levels for safety-related objectives.


\subsection{The Research Question}
What is the relationship between the mass of a damped spring-block oscillator {23.42g, 35.38g, 52.48g, 69.72g, 93.47g} and the damping ratio?

\subsection{Background Information}
An ideal and rather theoretical spring-mass system oscillates indefinitely, producing an ongoing sine or cosine curve. In reality, there will be a damping force that can be as minimally consequential as air resistance or as observable as viscous drag in a liquid. Anyway, energy is dissipated to the surroundings and hence the amplitude will gradually decrease until the oscillation stops. The extent to which the viscous drag force, modeled by Stoke's law, diminishes the oscillatory motion submerged in water depends on the mass of the oscillator. This investigation delves into the relationship between the independent variable of mass $m$, and the dependent variable of the damping ratio $\zeta$. The mass set used in the system consists of spheres of equal radii but different densities.

Stoke's law determines the drag force acting upon an object traveling through a fluid. It is proportional to the object's velocity and is given by the following per the Physics Data Booklet, in Newtons
$$F_d = 6\pi r \eta v $$
This law holds iff. the object speed is low such that the flow to be \textit{laminar}, and the object is \textit{spherical}.

\begin{assumption}
  \label{as:1}
  The damping force is the viscous drag force.
\end{assumption}

Moreover, the motion of the oscillator is modeled by the following differential equation, with the damping force proportional to velocity \parencite{miller_2004_13}:
\begin{equation}
  \label{eq:1}
  m\ddot{x}(t) + b\dot{x}(t) + kx(t) = 0
\end{equation}
The damping ratio $\zeta$ is defined as the ratio of the damping coefficient to the critical damping coefficient --- the damping coefficient when the system returns to equilibrium without completing an oscillation. That is $$\zeta = \frac{b}{c_c} = \frac{b}{2mw_0}$$
where $w_0$ is the natural frequency of the system
\begin{align*}
  w_0            & = \sqrt{\frac{k}{m}}   \\
  \implies \zeta & = \frac{b}{2\sqrt{mk}}
\end{align*}

\subsection{Hypothesis}

Suppose that the constant of proportionality of the damping force with speed is given by $b = 6\pi r \eta$, per Stoke's law. Then
\begin{align*}
  6\pi r \eta            & = 2\zeta \sqrt{mk}            \\
  9\pi^2 r^2 \eta^2      & = \zeta^2mk                   \\
  \frac{\zeta^2}{m^{-1}} & = \frac{9\pi^2 r^2 \eta^2}{k}
\end{align*}
\begin{proposition}
  \label{prop:1}
  For a fixed sphere radius $r$ and viscosity $\eta$, \begin{equation}\label{eq:2}
    \zeta^2 \propto m^{-1}
  \end{equation}
  Equivalently, the graph $\zeta^2/m^{-1}$ is ideally a straight line, although not necessarily passing through the origin, as some form of systematic errors are expected.
\end{proposition}

\cref{prop:1} has a few dependencies that may affect its accuracy. It relies on \cref{as:1}, which, in turn, requires that the flow is laminar. This is a reasonable assumption given that the spheres are relatively small, and the speed is low; the potential for turbulence is insignificant and even if present, the impact is minimal. Furthermore, the buoyancy force on the object
$$F_b = \rho g V$$
is always present but is not modeled in the differential equation at \cref{eq:1}. Notwithstanding, the absence of considerations of buoyancy has negligible impact on precision. The setup involves spheres of radius $2.5\si{\cm}$ and water of density $997\si{\kg\per\m\cubed}$, then
$$F_b \approx 10 \times 1000 \times \frac{4}{3}\pi (2.5\times 10^{-2})^3 < 1\si{\N}$$
\underline{Null hypothesis ($H_0$)}: There is no relationship between the mass and the damping ratio.
\underline{Alternative hypothesis ($H_1$)}: There is a relationship between the mass and the damping ratio.
\pagebreak

\section{Research Design}

\subsection{Variables}
\begin{center}
  \begin{tabular}{|p{0.15\textwidth}|p{0.15\textwidth}|p{0.3\textwidth}|p{0.3\textwidth}|}
    \hline
    \multicolumn{2}{|c}{\thcolor Variables} & \multicolumn{1}{|c}{\thcolor Explanation} & \multicolumn{1}{|c|}{\thcolor Measurement}                                                                                                                                                                                                                              \\  \hline
    Independent                             & Mass $m$                                  & The masses of the spherical metal bobs                                                                                                                                           & Measured using an electric balance, to two d.p. with uncertainty $\pm 0.005 \si{\g}$ \\
    \hline
    Dependent                               & Peak Amplitude                            & At least 5 consecutive peaks, if the oscillation does not stop before the 5th peak, will have their heights measured. These datasets are used then to compute the damping ratio. & Ruler, with uncertainty $\pm 0.0005\si{\m}$                                          \\
    \hline
  \end{tabular}
  \label{tab:1}
  \captionof{table}{Variables of investigation}
\end{center}

\begin{center}
  \begin{tabular}{|p{0.3\textwidth}|p{0.3\textwidth}|p{0.3
    \textwidth}|}

    \hline
    \multicolumn{3}{|c|}{\thcolor Control Variables}                                                                                                                                                                                                                                     \\ \hline
    \multicolumn{1}{|c|}{Variable} & \multicolumn{1}{|c|}{Rationale}                                     & \multicolumn{1}{|c|}{Means of Control}                                                                                                                                        \\ \hline
    Liquid viscosity               & To keep the drag coefficient per Stoke's Law constant               & Using \textbf{water from the same source} throughout the experiment, and keep the \textbf{room temperature} constant by turning off the air conditioning. Unknown uncertainty \\ \hline
    Radius of the sphere           & This is so that the drag force, per Stoke's law, is constant        & Using a set of spherical masses of the same diameter                                                                                                                          \\ \hline
    Spring constant                & This is so that the natural frequency of the system is constant     & Using the same spring throughout the experiment. Read from the label of on the box.                                                                                           \\ \hline
    Initial amplitude, $A_0$       & So that the damping ratio is not affected by the initial conditions & Pulling the spring down by equal distances for every mass.  Measured by a ruler and hence has uncertainty $\pm 0.0005\si{\m}$                                                 \\ \hline
    Damping coefficient            & Linearizes relationship between damping ratio and mass              & Previous means of control keep fluid viscosity and object dimensions constant, so the damping coefficient, which depend on these, will be constant.                           \\
    \hline
  \end{tabular}
  \label{tab:2}
  \captionof{table}{Control variables}
\end{center}

\subsection{Apparatus and Materials}
\img{figs/setup.png}{0.8}{The experimental setup}{setup}
\begin{outline}[enumerate]
  \1 Ruler ($\pm 0.0005\si{\m}$)
  \1 Electric balance ($\pm 0.005\si{\g}$)
  \1 Spring with spring constant $k = ???\si{\N\per\m}$
  \1 Mass set with five spherical masses of diameter $???\si{\cm}$, and masses
  \2 $m_1 = ???\si{\g}$
  \1 Beaker ([dimensions here])
  \1 Tap water ([volume here])
  \1 Camera --- keeps a record for future reference
  \1 Thermometer --- to ensure that the liquid stays at a constant temperature
  \1 Retort stand
  \1 Clamp
\end{outline}

\subsection{Methodology}

\begin{outline}[enumerate]
  \1 Take a piece of metal wire, and cut into segments of $3\si{\cm}$ in length. [\textcolor{orange}{may change to scotch tape}]
  \1 Arrange the masses in increasing order, i.e. $m_1 < m_2 < ... < m_n$, where $n$ is the number of masses in the mass set.
  \1 Fill in the beaker with $A\si{\mL}$ of tap water
  \1 Measure the temperature of the water, and record as $T_0$
  \1 Attach the clamp to the retort stand.
  \1 Lean the ruler vertically against the retort stand.
  \1 Place the beaker directly under the clamp such that the clamp lies at the center of the beaker in bird's eye view.
  \1 For $i$ from 1 to $n$ inclusive, repeat the following
  \2 Take the $i$th mass, $m_i$, and attach its hook to the spring.
  \2 Take a piece of metal wire, wrap it as many times as possible across the diagonal (explanation to be improved), fix and reinforce the connection between the mass and the spring.
  \2 Attach the spring-mass pair to the clamp and let the mass hang freely.
  \2 Adjust the height of the clamp so that the mass is submerged in the water.
  \2 Measure the height of the mass from the surface of the desk, and record as $h_0$
  \2 Adjust the vertical camera position such that it levels with the reading $(h_0 - \dfrac{A_0}{2}) \si{\cm}$ to minimize parallax error of the camera.
  \2 For $j$ from 1 to 3 inclusive, repeat the following
  \3 Pull the mass down by $A_0\,\si{\cm}$ by reading $h_0 - A_0$ from the ruler at eye level.
  \3 Start the camera recording.
  \3 Release the mass.
  \3 Record the clip until the 5th negative peak is reached.
  \3 Save the clip as \verb|Mass_i_Trial_j.mp4|
  \3 Open the clip in a video viewer.
  \3 Identify the first 3 negative peaks, one in each cycle.
  \3 Record the respective heights above the desk as $\{h_k\}^{5}_{k=1}$.
  \3 Then the corresponding peak amplitudes are $A_k = h_0 - h_{k}$
  \2 Take the average values across the three trials for each peak amplitude.
\end{outline}

\subsection{Preliminary Trials}

\subsection{Risk Assessment}

\begin{center}
  \begin{tabular}{|P{0.19\textwidth}|p{0.71\textwidth}|}
    \hline
    \thcolor Consideration & \multicolumn{1}{c|}{\thcolor {Relevance and Mitigation}         }                                                                                                                                                                                                                                 \\ \hline
    Safety                 & All valuable assets are kept away from the experimental setup to prevent accidental detachments of the spring-mass system leading to the mass being thrown out and causing damage. Moreover, the amplitude was kept low to prevent fierce movements that can break the beaker or computer camera. \\ \hline
    Ethical                & No use of animals or human bodies involved. However, the recording selects the location and angle that avoids the storage and exposure of any personal information or objects of the lab owner, ensuring maximum privacy protection.                                                              \\ \hline
    Environmental          & The entire experiment aims to reuse the same water throughout to minimize water wastage.    Furthermore, the spring was carefully pulled to prevent it from passing its elastic limit and become unusable for future experiments.                                                                 \\ \hline
  \end{tabular}
  \label{tab:3}
  \captionof{table}{Risk Assessment}
\end{center}

\section{Results}

\subsection{Raw Data}

\subsubsection{Qualitative Data}
During the entire experiment, there is no spillage of water, which means that the volume is kept constant. Moreover, the spring is always able to return to its original length without any permanent deformation, indicating that the spring is not stretched beyond its elastic limit and that Hooke's Law applies throughout.

\subsubsection{Quantitative Data}

Let $m$ denote the mass of a bob, and $A_n$ denote the $n$-th peak height.

\begin{table}[htbp]
  \begin{center}
    \bgroup
    \renewcommand{\arraystretch}{1.5}
    \resizebox{0.98\textwidth}{!}{%
      \begin{tabular}{|m{0.076\textwidth}|*{16}{m{0.0625\textwidth}|}}
        \hline
        \multicolumn{1}{|m{0.076\textwidth}|}{\thcolor {$m \pm 0.005$ ($\si{\g}$)}} & \multicolumn{5}{l|}{\thcolor \Large$A_1 \pm 0.05$ ($\si{\cm}$)} & \multicolumn{5}{l|}{\thcolor \Large$A_2 \pm 0.05$ ($\si{\cm}$)} & \multicolumn{5}{l|}{\thcolor \Large$A_3 \pm 0.05$ ($\si{\cm}$)}                                    \\ \hline
        23.42                                                                       &                                                                 &                                                                 &                                                                 &  &  &  &  &  &  &  &  &  &  &  & \\ \hline
        35.38                                                                       &                                                                 &                                                                 &                                                                 &  &  &  &  &  &  &  &  &  &  &  & \\ \hline
        52.48                                                                       &                                                                 &                                                                 &                                                                 &  &  &  &  &  &  &  &  &  &  &  & \\ \hline
        69.72                                                                       &                                                                 &                                                                 &                                                                 &  &  &  &  &  &  &  &  &  &  &  & \\ \hline
        93.47                                                                       &                                                                 &                                                                 &                                                                 &  &  &  &  &  &  &  &  &  &  &  & \\ \hline
      \end{tabular}}
    \label{tab:4}
    \captionof{table}{Raw data of peak amplitudes}
    \egroup
  \end{center}
\end{table}

\pagebreak

\subsection{Data Processing}

\subsubsection{Transformation}

Using the \textit{logarithmic decrement} method per \Citeauthor{inman_2008_engineering} (\citeyear{inman_2008_engineering})
$$\delta = \ln\paren{\frac{A_n}{A_{n+1}}} \text{ and } \delta = \frac{2\pi \zeta}{\sqrt{1 - \zeta^2}}$$ where $A_n$ and $A_{n + 1}$ denote any pair of consecutive peaks' heights.

\noindent It then follows that
\begin{align*}
  \delta^2                   & = \frac{4\pi^2 \zeta^2}{1 - \zeta^2}      \\
  \delta^2 - \delta^2\zeta^2 & = 4\pi^2\zeta^2                           \\
  \delta^2                   & = \zeta^2\paren{4\pi^2 + \delta^2}        \\
  \zeta                      & = \frac{\delta}{\sqrt{4\pi^2 + \delta^2}}
\end{align*}

This step is encapsulated in the following Python snippet that belongs to an overall script for processing the data and generating the graph.

\img{figs/code/damping.png}{0.8}{Calculating the damping ratio with pandas in Python}{damping_code}
\noindent Note that $\overline{A_n}$ denotes the average value of the $n$-th peak across the five trials.

\begin{table}[htbp]
  \begin{center}
    \bgroup
    \renewcommand{\arraystretch}{1.5}
    \begin{tabular}{|P{0.16\textwidth}|*{3}{P{0.12\textwidth}|}P{0.145\textwidth}|P{0.145\textwidth}|}
      \hline
      \thcolor {$m \pm 0.005$ ($\si{\g}$)} & \thcolor $\overline{A_1}$ ($\si{\cm}$) & \thcolor $\overline{A_2}$ ($\si{\cm}$) & \thcolor $\overline{A_3}$ ($\si{\cm}$) & \thcolor $\overline{\delta}$ (unitless) & \thcolor $\zeta$ (unitless) \\ \hline
      23.42                                &                                        &                                        &                                        &                                         &                             \\ \hline
      35.38                                &                                        &                                        &                                        &                                         &                             \\ \hline
      52.48                                &                                        &                                        &                                        &                                         &                             \\ \hline
      69.72                                &                                        &                                        &                                        &                                         &                             \\ \hline
      93.47                                &                                        &                                        &                                        &                                         &                             \\ \hline
    \end{tabular}
    \label{tab:5}
    \captionof{table}{Processed data}
    \egroup
  \end{center}
\end{table}

\subsubsection{Uncertainty Analysis}

\subsection{Analysis and Interpretation}

\section{Conclusion}

\subsection{Evaluation}

\subsection{Extensibility}

\pagebreak

\printbibliography

\end{document}