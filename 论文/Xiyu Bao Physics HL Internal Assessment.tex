\documentclass[a4paper,12pt]{article}
\usepackage{setspace}
\doublespacing
\usepackage[backend=biber,style=apa]{biblatex}
\addbibresource{Physics.bib}
\usepackage{sectsty}
\usepackage{siunitx}
\usepackage{graphicx}
\usepackage[a4paper, total={3in, 9in}, textwidth=16cm,bottom=1in,top=1.4in]{geometry}
\usepackage{xcolor}
\usepackage{amsmath}
\usepackage{esvect}
\usepackage{amsthm}
\usepackage{hyperref}
\usepackage{array} % for defining a new column type
\usepackage{varwidth} %for the varwidth minipage environment
\usepackage{float}
\usepackage{amssymb}
\usepackage{outlines}
\usepackage{caption}
\usepackage{subcaption}
\usepackage{esdiff}
\usepackage{setspace}
\newtheorem{lemma}{Lemma}
\newtheorem{proposition}{Proposition}
\newtheorem{assumption}{Assumption}
\doublespacing
\newcommand{\RNum}[1]{\uppercase\expandafter{\romannumeral #1\relax}}
\let\oldsi\si
\renewcommand{\si}[1]{\oldsi[per-mode=reciprocal-positive-first]{#1}}
\usepackage{enumitem}
\newcommand{\subtitle}[1]{%
  \posttitle{%
    \par\end{center}
    \begin{center}\large#1\end{center}
    \vskip0.5em}%
}
\newcommand{\degsym}{^{\circ}}
\newcommand{\Mod}[1]{\ (\mathrm{mod}\ #1)}
\usepackage{hyperref}
\hypersetup{
  colorlinks,
  citecolor=black,
  filecolor=black,
  linkcolor=black,
  urlcolor=black
}
\newcommand{\lb}{\\[8pt]}
\newenvironment*{cell}[1][]{\begin{tabular}[c]{@{}c@{}}}{\end{tabular}}
\newcommand{\img}[4]{\begin{center}
  \begin{figure}[H]
    \centering
    \includegraphics[width=#2\textwidth]{#1}
    \caption{#3}
    \label{fig:#4}
  \end{figure}
\end{center}}
\newcommand{\doubleimg}[4]{\begin{center}
  \begin{figure}[H]
    \centering
    \begin{subfigure}{.45\textwidth}
      \centering
      \includegraphics[width=1\linewidth]{#1}
      \caption{#2}
      \label{fig:sub1}
    \end{subfigure}
    \begin{subfigure}{.45\textwidth}
      \centering
      \includegraphics[width=1\linewidth]{#3}
      \caption{#4}
      \label{fig:sub2}
    \end{subfigure}
  \end{figure}
\end{center}}
\usepackage{fancyhdr}
\fancyfoot{}
\newcommand{\vect}[3]{\begin{bmatrix}
  #1 \\
  #2 \\
  #3
\end{bmatrix}}
\fancypagestyle{fancy}{\fancyfoot[R]{\vspace*{1.5\baselineskip}\thepage}}
\renewcommand{\contentsname}{Table of Contents}
\newcommand{\angled}[1]{\langle{#1}\rangle}
\newcommand{\paren}[1]{\left(#1\right)}
\newcommand{\sqb}[1]{\left[#1\right]}
\newcommand{\coord}[3]{\angled{#1,\, #2,\, #3}}
\newcommand{\pair}[2]{\paren{#1,\, #2}}
\usepackage{cleveref}
\crefname{lemma}{Lemma}{Lemmas}
\crefname{assumption}{Assumption}{Assumptions}
\crefname{proposition}{Proposition}{Propositions}
\setlength{\headheight}{15pt}
\newcolumntype{P}[1]{>{\centering\arraybackslash}p{#1}}

\begin{document}


\pagenumbering{arabic}
\pagestyle{fancy}


\begin{titlepage}
  \begin{center}
    \vspace*{3cm}

    {\textbf{\Large{What is the relationship between the mass of a damped spring-block oscillator and the damping ratio?}}}

    \vspace{1cm}
    \large{Physics HL}\\
    \large{Internal Assessment}


    \vfill

    \vspace{1.5cm}

    Word count: TBD

  \end{center}
\end{titlepage}
\pagebreak
\tableofcontents
\pagebreak

\clearpage
\setcounter{page}{1}
\addtocontents{toc}{\protect\thispagestyle{empty}}

\section{Introduction}
This essay extends the investigation of simple harmonic motion by studying the damping force of a damped oscillator submerged in water, aiming to scrutinize the relationship between the mass of the block and the damping ratio. By scrutinizing this relationship, the study aims to offer valuable insights that can inform the design of systems seeking to optimize damping levels for safety-related objectives.


\subsection{The Research Question}
What is the relationship between the mass of a damped spring-block oscillator and the damping ratio?

\subsection{Background Information}
An ideal and rather theoretical spring-mass system oscillates indefinitely, producing an ongoing sine or cosine curve. In reality, there will be a damping force that can be as minimally consequential as air resistance or as observable as viscous drag in a liquid. Anyway, energy is dissipated to the surroundings and hence the amplitude will gradually decrease until the oscillation stops.

The extent to which the viscous drag force, modeled by Stoke's law, diminishes the oscillatory motion submerged in water depends on the mass of the oscillator. This investigation delves into the relationship between the independent variable of mass $m$, and the dependent variable of the damping ratio $\zeta$. The mass set used in the system consists of spheres of equal radii but different densities.

Stoke's law determines the drag force acting upon an object traveling through a fluid. It is proportional to the object's velocity and is given by the following per the Physics Data Booklet, in Newtons
$$F_d = 6\pi r \eta v $$
This law holds iff. the object speed is low such that the flow to be \textit{laminar}, and the object is \textit{spherical}.

\begin{assumption}
  \label{as:1}
  The damping force is the viscous drag force.
\end{assumption}

Moreover, the motion of the oscillator is modeled by the following differential equation, with the damping force proportional to velocity \parencite{miller_2004_13}:
\begin{equation}
  \label{eq:1}
  m\ddot{x}(t) + b\dot{x}(t) + kx(t) = 0
\end{equation}
The damping ratio $\zeta$ is defined as the ratio of the damping coefficient to the critical damping coefficient --- the damping coefficient when the system returns to equilibrium without completing an oscillation. That is $$\zeta = \frac{b}{c_c} = \frac{b}{2mw_0}$$
where $w_0$ is the natural frequency of the system
\begin{align*}
  w_0            & = \sqrt{\frac{k}{m}}   \\
  \implies \zeta & = \frac{b}{2\sqrt{mk}}
\end{align*}

\subsection{Hypothesis}

Suppose that the constant of proportionality of the damping force with speed is given by $b = 6\pi r \eta$, per Stoke's law. Then
\begin{align*}
  6\pi r \eta            & = 2\zeta \sqrt{mk}            \\
  9\pi^2 r^2 \eta^2      & = \zeta^2mk                   \\
  \frac{\zeta^2}{m^{-1}} & = \frac{9\pi^2 r^2 \eta^2}{k}
\end{align*}
\begin{proposition}
  \label{prop:1}
  For a fixed sphere radius $r$ and viscosity $\eta$, \begin{equation}\label{eq:2}
    \zeta^2 \propto m^{-1}
  \end{equation}
  Equivalently, the graph $\zeta^2/m^{-1}$ is ideally a straight line, although not necessarily passing through the origin, as some form of systematic errors are expected.
\end{proposition}

\cref{prop:1} has a few dependencies that may affect its accuracy. It relies on \cref{as:1}, which, in turn, requires that the flow is laminar. This is a reasonable assumption given that the spheres are relatively small, and the speed is low; the potential for turbulence is insignificant and even if present, the impact is minimal. Furthermore, the buoyancy force on the object
$$F_b = \rho g V$$
is always present but is not modeled in the differential equation at \cref{eq:1}. Notwithstanding, the absence of considerations of buoyancy has negligible impact on precision. The setup involves spheres of radius $2.5\si{\cm}$ and water of density $997\si{\kg\per\m\cubed}$, then
$$F_b \approx 10 \times 1000 \times \frac{4}{3}\pi (2.5\times 10^{-2})^3 < 1\si{\N}$$
\pagebreak

\section{Research Design}

\subsection{Variables}
\begin{center}
  \begin{tabular}{|p{0.15\textwidth}|p{0.15\textwidth}|p{0.3\textwidth}|p{0.3\textwidth}|}
    \hline
    \multicolumn{2}{|c}{Variables} & \multicolumn{1}{|c}{Explanation} & \multicolumn{1}{|c|}{Measurement}                                                                                                                                                                                                                                       \\  \hline
    Independent                    & Mass $m$                         & The masses of the spherical metal bobs                                                                                                                                           & Measured using an electric balance, to two d.p. with uncertainty $\pm 0.005 \si{\g}$ \\
    \hline
    Dependent                      & Peak Amplitude                   & At least 5 consecutive peaks, if the oscillation does not stop before the 5th peak, will have their heights measured. These datasets are used then to compute the damping ratio. & Ruler, with uncertainty $0.0005\si{\m}$                                              \\
    \hline
  \end{tabular}
  \label{tab:1}
  \captionof{table}{Variables of investigation}
\end{center}

\begin{center}
  \begin{tabular}{|P{0.3\textwidth}|p{0.3\textwidth}|p{0.3
    \textwidth}|}

    \hline
    \multicolumn{3}{|c|}{Control Variables}                                                                                                                                                                                                               \\ \hline
    Variable             & Rationale                                                           & Means of Control                                                                                                                                         \\ \hline
    Liquid viscosity     & To keep the drag coefficient per Stoke's Law constant               & Using \textbf{water from the same source} throughout the experiment, and keep the \textbf{room temperature} constant by turning off the air conditioning \\ \hline
    Radius of the sphere & This is so that the drag force, per Stoke's law, is constant        & Using a set of spherical masses of the same diameter                                                                                                     \\ \hline
    Spring constant      & This is so that the natural frequency of the system is constant     & Using the same spring throughout the experiment                                                                                                          \\ \hline
    Initial amplitude    & So that the damping ratio is not affected by the initial conditions & Pulling the spring down by equal distances for every mass.                                                                                               \\ \hline
    Damping coefficient  & Linearizes relationship between damping ratio and mass              & Previous means of control keep fluid viscosity and object dimensions constant, so the damping coefficient, which depend on these, will be constant.      \\
    \hline
  \end{tabular}
  \label{tab:2}
  \captionof{table}{Control variables}
\end{center}

\subsection{Apparatus and Materials}
\img{figs/setup.png}{0.8}{The experimental setup}{setup}

\subsection{Methodology}

\subsection{Risk Assessment}

\section{Results}

\subsection{Raw Data}

\subsection{Data Processing}

\subsection{Analysis and Interpretation}

\section{Conclusion}

\subsection{Evaluation}

\subsection{Extensibility}

\pagebreak

\printbibliography

\end{document}